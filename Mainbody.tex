\documentclass{article}
\usepackage[UTF8]{ctex}
\begin{document}
\title{兴趣使然的博弈论笔记(这篇tex是我在大二时整的烂活,但是因为是我在github提交的第一个原创文件所以暂时留着它,现在这个仓库用作记录我学习cpp的心得)}
\maketitle
\tableofcontents
\section{博弈论简介} \footnote{此笔记的参考教材主要是张维迎的《博弈论与信息经济学》,此外还有Gibbons的《博弈论基础》}
\section{NE}
\subsection{中央与地方基础设施投资博弈}
\subsection{单阶段同时出价模型}
\section{MNE}
\subsection{社会福利博弈}
\section{SPE}
\subsection{通货膨胀政策的不一致性}
\subsection{轮流出价的讨价还价模型}
\section{PBE}
\subsection{劳动市场信号博弈(Spence)}\footnote{"能力,教育水平与工资"}
\section{信息经济学}
\subsection{简单离散委托—代理模型}
\subsection{柠檬市场}
\subsection{商品市场信号发送}
\subsection{菜单定价}

%正文部分

\section{博弈论简介} \footnote{此笔记的参考教材主要是张维迎的《博弈论与信息经济学》,此外还有Gibbons的《博弈论基础》}
\section{NE}
\subsection{中央与地方基础设施投资博弈}

\subsection{单阶段同时出价模型}
\section{MNE}
\subsection{社会福利博弈}
\section{SPE}
\subsection{通货膨胀政策的不一致性}
\subsection{轮流出价的讨价还价模型}
\section{PBE}
\subsection{劳动市场信号博弈(Spence)}\footnote{"能力,教育水平与工资"}
符号:$q,t,L,H,e,c(t,e),w(e),f(t,e),y(t,e),U(w,e),e_p$\\\\

要素:博弈树,q=mu(H|e)为企业的信念,收益,t(分为L,H两种)为工人类型,e为教育程度\emph{即工人的策略},\
c(t,e)为接受教育的成本,f(t,e)为工人为企业创造的产出,w(e)为工资合同\emph{即企业的策略},\
不同种类工人的无差异曲线(陡峭程度不同,只由工资和教育程度决定,教育程度增加,需要工资相应增加(增加值为两点纵坐标差值)\
才能保持效用不变) \footnote{列出这些的原因在于:上课时我(和大部分还没躺平的同学)经常因为不知道某个符号代表什么意思而跟不上进度}\\\\

结论:即不同情况下的均衡(即不同情况下工人和企业的策略)\emph{“不同情况”指: 1企业的
信念以及基于此信念设计的工资合同 2不同类型工人无差异曲线的分布情况}\\\\
\begin{center}
    \zihao{4}
    \textbf{\emph{案例:原模型在现实中的应用}\\\\}
    \end{center}
\footnote{我觉得先举一个接地气的例子而非直接看原教材中的例子更能激发读者的学习兴致}

先分析一个现实中的问题:\\\\
\begin{center}
    \zihao{4}
    \textbf{\emph{原模型的推理过程}}\\\\
    \end{center}
原模型的推理过程(按照我的理解):博弈树-本模型的假设-完全信息情形-不完全信息时的两种情况-混同均衡-分离均衡\\\\

博弈树如图\\

本模型的5个前提假设\\完全信息情形"均衡点的形成过程:1最大化U(w,e)=y(t,e)-c(t,e)得到e*(t),2由e*(t)得到w*(t),3过此点的无差异曲线"\\

不完全信息时的两种情况"有动机模仿,不想模仿"\\

混同均衡"均衡路径与非均衡路径,企业观察后的推断不变"

推断不变决定工资合同,还必须描述:非均衡的推断,最优反应,特例工人的选择,图像及其描述,其他混同PBE\\
\\分离均衡\\\\


\footnote{课后自学ppt很难看懂;如果每节课前没花几小时的预习,
那么课上也很难听懂;课后找到Gibbons的教材才基本搞懂这个模型。}

\begin{center}
    \zihao{4}
    \textbf{\emph{一些重要问题\footnote{可以用于检验你是否真正理解了这个模型}}}\\\\
    \end{center}

在每个情形下,工人和企业面临的选择是什么? 工人:最大化w(e)-c(t,e) 企业:最大化y(t,e)-w(e)\\
工人和企业最终采取的策略是什么?\\
这个博弈的前提情景、假设、最后的结果是什么?\\
如果我在现实中看到一个符合这个模型的所
